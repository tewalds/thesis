
Here are general game playing concepts and definitions:

\begin{description}
\item[Perfect Information] A game has the property of perfect information when both players know the full state of the game.
\item[Stochastic] A game is stochastic if it has elements of randomness, such as dice rolls. Backgammon is stochastic while Chess, Hex and Havannah are not.
\item[Zero Sum Games] A zero sum game has the property that one players gain is the other players loss. A draw is still possible, but no move can help both players, so there is no incentive to cooperate.
\item[State] A state is a full description of a board position. It includes the locations of all the pieces, and any other relevant information. In chess, the state would include whether each king can castle, and whether a pawn can capture \textit{en passant}. In games where repeated moves are not allowed, the full game history may be included in the state. Occasionally a simplified version of the state is used when speed is more important than accuracy.
\item[Move] A move is a distinct action by one of the players leading from one state to another state. In games with multi-part moves, such as Amazons where each move consists of a movement plus shooting an arrow, the pair of actions would be considered a single move. There are usually multiple moves available from each state, but usually only one can be chosen per turn.
\item[History] All moves leading from some starting position, usually the beginning of the game, to the current state.
\item[Branching Factor] is the number of moves available to each player on average. This depends on the rules of the game, board size, pieces in play and the stage of the game. This can be as low as 1 for forced moves, or very high, such as in the hundreds or thousands for Amazons or Arimaa.
\item[Game Tree] Games can be represented as a game tree. Each position in the game is a node, and each move is an edge in the graph connecting the position before the move to the position after the move. When there are multiple paths to a position, it can be represented as separate nodes, leading to a tree, or combined as a single node, leading to a directed acyclic graph (DAG). Some games have loops, where a position can be reached multiple times in a single game, leading to a directed graph.
\item[Root Node] The root node is the highest node in the tree. It has no parents, and there is exactly one of them in any tree.
\item[Leaf Node] A leaf node is any node that has no children.
\item[Node Value] Each node has an associated outcome or expected outcome associated with it. Terminal positions are positions where one of the players has won or it is a draw, have an exact value such as win, loss or draw, or a score to show how much a player won by.
\item[Heuristic] A heuristic function takes a position and returns a value associated with the position. This value often represents the likelihood of winning from that position, but can also be just an abstract number that can be compared against other values to order nodes or moves.
\item[State Space] is the number of unique reachable states in the game.
\item[Game Complexity] is the size of the state space, sometimes taking transpositions into account. This can either be the number of unique positions or the number of possible games.
\item[Minimax] In 2-player games, each player attempts to win at the expense of the other player. To do so, each player attempts to minimize the opponent's gain while maximizing their own gain. To win, a player must at at least one winning move, but to lose all moves must be losing moves.
\item[Minimax Backup] Given a node N who's children all have known values, N's value is equal to the value of the most favourable child for the current player.
\item[Minimax Value] The value of a node given that both players play perfectly according to Minimax.
\item[Transposition] One state with multiple histories. If moves A-X-B leads to the same position and state as B-X-A, they are transpositions. They are the same state, so they will have the same minimax value, and should not be searched twice.
\item[Hash Value] A hash value is a representative number of a state used to detect transpositions. Transpositions all have the same hash value, but different states have different hash values. Often collisions are possible so two states that aren't a transposition have the same hash value, but this is very rare as large hash values (usually 64 bit unsigned integers) are used.
\item[Zobrist Hash] In some search spaces a hash value can be built up incrementally by XORing a random string associated with each move against a previous hash value.
\item[Depth First Search] In a depth first search (DFS), nodes are considered in a depth-first way. The full subtree of a node will be explored before any of its siblings will be explored. This is very memory efficient since it only needs to store the nodes along the path from the root to the current node, but leaves many nodes near the root unexplored for long periods of time.
\item[Breadth First Search] In a breadth first search, all nodes at a specific depth will be considered before any nodes at a deeper depth, in increasing depth. This is very memory intensive as all nodes up to the specified depth must be kept in memory.
\item[Best First Search] In a best first search nodes are explored in order of their heuristic value. Promising nodes are explored before less promising nodes. This is very memory intensive as all nodes explored to date must be kept in memory.
\item[Anytime algorithm] An anytime algorithm can return an answer at any point of execution, but continues to run to provide a more accurate and potentially better answer.
\end{description}
